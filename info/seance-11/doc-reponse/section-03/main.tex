\q{Soit $S_d$ la variable aléatoire donnant la loi de la somme de $d$ dés, montrer que :}
\[
  \mathbb{P}(S_d=k) = \frac{1}{6}\sum\limits_{j=1}^{6}\mathbb{P}(S_{d-1}=k-j)
\]

\bigskip

Soit $\Omega$ l'univers. Alors, $X(\Omega) = [\![d, 6d]\!]$.\\
On pose $Z$ la variable aléatoire donnant la loi du dernier dé. \\
Alors, $((Z=j))_{1\leq j\leq 6}$ est un système complet d'évènements.
Ainsi, pour un évenement $E$ quelconque, la formule des probabilités totales donne :
\[
  \mathbb{P}(E) = \sum\limits_{j=1}^{6}\mathbb{P}(E | Z = j)\mathbb{P}(Z=j)
\]
Ainsi :

\[
  \begin{array}{lrcl}
                & \mathbb{P}(S_d=k) & = & \frac{1}{6}\sum\limits_{j=1}^{6}\mathbb{P}(S_d=k | Z = j)  \\
    \Rightarrow & \mathbb{P}(S_d=k) & = & \frac{1}{6}\sum\limits_{j=1}^{6}\mathbb{P}(S_{d-1}= k - j) \\
  \end{array}
\]
On retrouve bien la formule précédente.
