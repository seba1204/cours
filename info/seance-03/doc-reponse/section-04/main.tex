\section{Notation Polonaise inverse}
\begin{enumerate}
      \item \q{Créer une fonction }
            \il{est_operateur}
            \q{ qui prend une chaîne de caractère en argument et renvoie }
            \il{True}
            \q{ si cette chaîne est }
            \il{"+"}\q{, }\il{"*"}\q{, }\il{"/"}\q{ ou }\il{"-"}\q{, }
            \il{False}\q{ sinon}
            \codeFromFile{section-04/q1.py}
      \item \q{Créer une fonction }\il{calcul}\q{ qui prend en argument un opérateut }
            \il{op}\q{ et deux opérandes }\il{a}\q{ et }\il{b}\q{ et qui renvoie la valeur de }\il{ a op b}\q{.}
            \codeFromFile{section-04/q2.py}
      \item \q{Définir une fonction }\il{liste_mots}\q{ qui prend en argument une chaîne de caractères et
                  retourne la liste des caractères non séparés par un espace.}
            \codeFromFile{section-04/q3.py}
      \item \q{En utilisant les deux fonctions précédentes, définir la fonction }\il{polonaise}\q{.}
            \codeFromFile{section-04/q4.py}
\end{enumerate}
