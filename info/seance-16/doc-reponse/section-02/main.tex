\section{Calculer une ligne}
\q{Construire }\il{points(n, pas, x, y)}\q{ qui renvoie une liste
$A = [A_i]_{i\in [\![o, n-1]\!]}$ contenant $n$ points $A_i$ sur la ligne de
niveau $C_k$ où $k = f(A_0)$ en commençant par $A_0 = [x,y]$ et où tous les
points suivants $A_i$ seront construits par récurrence de la manière suivante :}
\begin{enumerate}[(a)]
  \item \q{En $A_i$ sur la ligne de niveau $C_0$ on calcule vecteur tangent
          unitaire $\vec{T}$ à partir de $\nabla f(A_i)$}
  \item \q{On calcule le point intermédiaire $B_i$ tel que
          $\overrightarrow{A_iB_i}=$ }\il{pas}\q{$\times\vec{T}$}
  \item \q{On calcule $A_{i+1}$ tel que $\overrightarrow{OA_{i+1}}=
            \overrightarrow{OB_i} + \lambda \nabla f(B_i)$ où $\lambda$
          est tel que, d'après la formule de Taylor-Young à l'ordre 1, le point
          $A_{i+1}$ soit sur la ligne de niveau $C_k$ où $k = f(A_0)$. La valeur
          de $\lambda$ est supposée être faible devant celle du }\il{pas}\q{.}
\end{enumerate}
Par exemple, pour 10 points avec un pas de 0.5 à partir de (1, 0) on trouve
\il{A = points(10, 0.5, 1, 0)} $\Rightarrow$ A = [
[1.0, 0.0],
[0.62, 0.32],
[0.21, 0.6],
[-0.24, 0.82],
[-0.71, 0.97],
[-1.19, 0.99],
[-1.43, 0.85],
[-1.44, 0.52],
[-1.16, 0.15],
[-0.8, -0.19]
]
\begin{dinglist}{111}
  \item Je crée le fichier \il{test/exceptedOutput.py} dans lequel je mets les
  résultats attendus pour tester ma fonction à la fin :
  \codeFromFileT{test/exceptedOutput.py}{section-02/qa.py}
  \item J'importe les résultats dans le fichier \il{main.py}
  \codeFromFileT{main.py}{section-02/qb.py}
  \item Je détermine $\lambda$ :
  On a : $f(A_{i+1}) = k$, donc d'après la formule de Taylor à l'ordre 1 :
  \[
    k = f(A_{i+1}) = f(B_i) + \nabla f(B_i)\cdot\overrightarrow{B_iA_{i+1}}
  \]
  Or : $\overrightarrow{B_iA_{i+1}} = \lambda \nabla f(B_i)$\\
  Donc : $\nabla f(B_i)\cdot\overrightarrow{B_iA_{i+1}} = \nabla f(B_i)\cdot\lambda
    \nabla f(B_i)$\\
  D'où :
  \begin{result}
    $\lambda = \frac{k-f(B_i)}{||\nabla f(B_i)||^2}$
  \end{result}
  \item Je code la fonction \il{points}
  \codeFromFileT{main.py}{section-02/qc.py}

  \item Je code un fonction \il{around} pour vérifier mes résultats :
  \codeFromFileT{helpers/help.py}{section-02/qd.py}

  \item Je teste tout ça :
  \codeFromFileT{main.py}{section-02/qe.py}
  Et le console affiche \il{True} !
\end{dinglist}
