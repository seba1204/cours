\section{Passage de paramètre}
\q{On donne les commandes suivantes :}
\codeFromFile{section-01/q1.py}
\medskip
\begin{enumerate}
  \item
        \q{Donner le contenu des variables }
        \il{p}\q{,}
        \il{s}\q{,}
        \il{q}\q{,}
        \il{r}\q{ et}
        \il{u}\q{ à la fin du programme.}

        \begin{dinglist}{111}
          \item La ligne 9 implique : \il{s = [5, 2, 3, 4]}
          \item La ligne 10 implique : \il{q = s}. En effet la fonction \il{f} modifie la variable globale \il{p} (qui lui est
          passée en paramètre) car \il{p} est une liste donc mutable, donc la ligne 5 modifie directement \il{p}.
          \item La ligne 20 implique : \il{r = [5, 3, 2, 1]} car la commande \il{p = p[::-1]} inverse \il{p}.
          \item La ligne 21 par contre implique : \il{u = p = [1, 2, 3, 4]}. En effet, cette fois-ci avant d'appliquer la
          commande \il{p[0] = 5}, on copie tout le contenu de \il{p} dans une nouvelle variable. Ainsi, la variable globale
          \il{p} (passée en paramètre de \il{g}) n'est pas modifiée, c'est cette nouvelle variable qui l'est.
        \end{dinglist}

        D'où :
        \begin{result}
          $
            \left\{
            \begin{array}{rcccl}
              \il{s} & = & \il{q} & = & \il{[5, 2, 3, 4]} \\
              \il{p} & = & \il{u} & = & \il{[1, 2, 3, 4]} \\
                     &   & \il{r} & = & \il{[5, 3, 2, 1]} \\
            \end{array}
            \right.
          $
        \end{result}

\end{enumerate}
