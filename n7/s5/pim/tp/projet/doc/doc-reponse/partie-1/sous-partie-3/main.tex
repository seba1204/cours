Voici un résumé du sujet. Pour plus de détail, se référer au \href{https://raw.githubusercontent.com/seba1204/cours/master/n7/s5/pim/tp/projet/doc/sujet.pdf}{sujet}.

Un fichier représentant le reseau nous est donné. Voici un exemple :

\lstinputlisting[caption=Fichier reseau exemple]{partie-1/sous-partie-3/exemple_fichier.txt}

La ligne 1 nous indique le nombre de sites (internet) composant le reseau (ici 6).

Chaque autre ligne est composé de deux nombres : 
\begin{itemize}
   \item le $1^{er}$ est le site référenceur (une de ses pages internet contient un lien vers le site référencé)
   \item le $2^{nd}$ est le site référencé
\end{itemize}

Par exemple la $2^{\grave{e}me}$ ligne nous dit que le site \lstinline{0} fait référence aux sites \lstinline{1} et \lstinline{2}.

Pour savoir quel est le site le plus populaire, il faut alors calculer le nombre de référencements.
On peut répertorier tous ces liens dans une matrice en mettant en ligne les référenceurs et en colonne les référencés :
\[
   \begin{bmatrix}
      0 & 1 & 1 & 0 & 0 & 0  \\
      0 & 0 & 0 & 0 & 0 & 0  \\
      1 & 1 & 0 & 0 & 1 & 0  \\
      0 & 0 & 0 & 0 & 1 & 1  \\
      0 & 0 & 0 & 1 & 0 & 1  \\
      0 & 0 & 0 & 1 & 0 & 0  \\
   \end{bmatrix}
\]

On retrouve bien à la $1^{\grave{e}re}$ ligne que le site \lstinline{0} fait référence aux sites \lstinline{1} et \lstinline{2}.

Après quelques calculs (c.f. sujet) on retrouve une matrice $G_{\alpha}$ de paramètre $\alpha \in [0,1]$ qui permet de nuancer la popularité d'un site, et d'avoir des résultats plus proches de la réalité.
Pour calculer les poids des sites $\pi$, il suffit alors de faire : 
\[
   \pi^{T}_{k+1} = \pi^{T}_{k} \cdot G
\]
avec $\pi$ le vecteur ligne contenant les poids des sites.

Après $M$ iterations, le vecteur $\pi_{M}$ est une bonne représentation de la popularité des sites dans la réalité.