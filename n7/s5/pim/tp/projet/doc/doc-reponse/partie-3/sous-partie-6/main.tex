\begin{dinglist}{111}
   \item \textbf{Une table de hachage.}
   $G$ doit maintenant être creuse, donc tous les $0$ ne sont pas pris en compte. Pour cela nous avons choisi d'implémenter une table de hachage :
   \begin{lstlisting}[caption=Type de \lstinline{Google_Creuse}]
-- Google_Creuse.ads
...
type Cellule is record 
   Index: Integer;
   Donnee: T_Donnee;
   Suivant: Ptr_Cellule;
end record;
type Ligne is record 
   Index: Integer;
   Cellule: Ptr_Cellule;
   Suivant: Ptr_Ligne;
end record;
type Th is array (Positive range <>) of Ptr_Ligne;
type Table_Hachage(Size: Integer) is tagged record
   Table : Th(1..Size);
end record;

...
-- Google.adb
declare
   G : Google_Creuse (A.Taille_Hachage);
   ...
      \end{lstlisting}
Nous avons donc un \lstinline{record Cellule} qui contient le coefficient de $G$. Tous les coefficients d'une ligne se rattachent l'un à l'autre (grâce à la propriété \lstinline{Suivant}).
Chaque ligne est  \lstinline{record} contenant une \textit{grappe} de coefficients. Ces lignes elles-mêmes se suivent de la même manière.
Enfin un tableau de taille relativement petite contient des lignes. La clé de hachage est l'index de la ligne. Ainsi, si la taille du tableau est $100$, et la taille du reseau est $1 000$, 
alors, la première case du tableau contiendra les lignes d'index $0, 100, 200, 300, ..., 900$.
   \item \textbf{Implémentation des fonctions.} Il est maintenant plus compliqué d'implémenter les fonctions que précédemment. Par exemple la fonction \lstinline{Inserer} doit insérer dans la bonne ligne si elle existe, et entre deux Cellules, si elles existent 
   la valeur voule...
   \item \textbf{Transposée.} Notons que pour le calcul des poids : $\pi^{T}_{k+1} = \pi^{T}_{k} \cdot G$, c'est la colonne de $G$ qui nous intéresse. C'est pourquoi, lors de la création de $G$,
   nous calculons réellement sa transposée.
\end{dinglist}