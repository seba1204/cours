\begin{dinglist}{111}
   \item \textbf{Un type \lstinline{array}.}
     Comme demandé explicitement dans le sujet : << \lstinline{Google_Naive} definit et manipule une matrice G sous la forme d’un tableau de reels à deux dimensions statiques >>.
     Nous somme donc obligé de faire un tableau de dimension $N \times N$ avec $N$ la taille du reseau. La dimension n'est pas à proprement parler \textit{statique}
   car nous avons voulu que la dimension du tableau varie en fonction de $N$ : 
   \begin{lstlisting}[caption=Type de \lstinline{Google_Naive}]
-- Google_Naive.ads
type G_N is array (Positive range <>, Positive range <>) of T_Float;
type Google_Naive(Size: Positive) is new Google with record
   Table : G_N(1..Size, 1..Size);
end record;
...
-- Google.adb
declare
   G : Google_Naive (A.Taille_Reseau);
   ...
      \end{lstlisting}
   \item \textbf{Implémentation des fonctions.} La surchage des fonctions de la classe mère \lstinline{Google} sont relativement simples : des boucles for imbriquées, et un accès aux données via \lstinline{Table(I,J)}.
   \item \textbf{Mémoire.} La taille utilisée par cette matrice est rapidement très grande ($N^{2}$). Il n'est donc pas possible de traiter des fichiers reseaux trop gros avec ce type de matrice.
   Nous avons codé une autre implémentation de la matrice << naïve >> en enregistrant son contenu au fur et à mesure dans un fichier, mais en la testant nous obtenions des fichiers de taille théorique dépassant les 300Go.
   (Nous n'avons pas attendu la fin de l'exécution.) Nous avons jugé inutile de continuer dans cette voie, et de laisser une erreur Stack Overflow si le reseau est trop grand.
   
\end{dinglist}