La gestion des arguments (paramètres de la commandline \lstinline{./pagerank [parametres]}) se fait dans le module \lstinline{helpers}.
\begin{dinglist}{111}
   \item \textbf{Les différents paramètres possibles.}
   \begin{itemize}
      \item \lstinline{-P} : (Optionnel) Permet de choisir l'implémentation creuse de Google. Si cet argument n'est pas présent, c'est l'implémentation naïve qui est choisie.
      \item \lstinline{-I} : (Optionnel) Doit être suivi d'un nombre entier positif. Permet de paramétrer le nombre d'iterations faites pour calculer les poids. Valeur par défaut : 150
      \item \lstinline{-A} : (Optionnel) Doit être d'un réel. Le paramètre doit être un réel compris entre 0 et 1. Permet d'indiquer le paramètre alpha du la matrice Google.
      \item \lstinline{-V} : (Optionnel) Active le mode verbeux. Si activé, toutes les étapes de calcul seront loguées dans la console.
      \item \lstinline{-H} : (Optionnel) Affiche cet aide.
      \item \lstinline{nomfichier.net} : (Obligatoire) La ligne de commande doit se terminer par le nom .net qui contient les liens entre les pages web. 
   \end{itemize}
   \item \textbf{Le type \lstinline{Args}.} Le type \lstinline{Args} est l'enregistrement suivant :
\begin{lstlisting}[caption=Le type \lstinline{Args}]
type T_Args is record 
   Nom_Fichier    : Unbounded_String;
   A              : Float    := 0.85;
   Max_Iter       : Integer  := 150;
   Est_Naif       : Boolean  := True;
   Aide_Demande   : Boolean  := False;
   Taille_Reseau  : Positive;
end record;
\end{lstlisting}
   \item \textbf{Lecture des arguments.} La fonction \lstinline{Lire_Arguments} se charge de lire les arguments un par un, et à l'aide d'un
   \lstinline{switch} détermine si un argument est présent ou non, et éventuellement sa valeur.
   \item \textbf{Mode verbeux.} La fonction \lstinline{Log(Message);} par exemple, n'affiche le message que si le mode verbeux à été passé en paramètre lors du lancement du programme. De même la fonction 
   \lstinline{Log_P(Progression, Maximum)} affiche une \textit{Progress Bar}.

\end{dinglist}