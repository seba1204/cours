% \documentclass{sebaClass}
% \title{TP 01}
% \author{PONT Sébastien}
% \date{29 janvier 2021}

% \newcommand{\SubTitle}{Points et segments}
% \newcommand{\Subject}{Technologie Objet}
% \newcommand{\Place}{ENSSEIHT}

\documentclass{article}
\usepackage[utf8]{inputenc}
\usepackage[english]{babel}

\usepackage{minted}

\begin{document}

% \maketitle

\section{Grand 1}

\begin{minted}{python}
  import numpy as np
      
  def incmatrix(genl1,genl2):
      m = len(genl1)
      n = len(genl2)
      M = None #to become the incidence matrix
      VT = np.zeros((n*m,1), int)  #dummy variable
      
      #compute the bitwise xor matrix
      M1 = bitxormatrix(genl1)
      M2 = np.triu(bitxormatrix(genl2),1) 
  
      for i in range(m-1):
          for j in range(i+1, m):
              [r,c] = np.where(M2 == M1[i,j])
              for k in range(len(r)):
                  VT[(i)*n + r[k]] = 1;
                  VT[(i)*n + c[k]] = 1;
                  VT[(j)*n + r[k]] = 1;
                  VT[(j)*n + c[k]] = 1;
                  
                  if M is None:
                      M = np.copy(VT)
                  else:
                      M = np.concatenate((M, VT), 1)
                  
                  VT = np.zeros((n*m,1), int)
      
      return M
  \end{minted}


\end{document}
