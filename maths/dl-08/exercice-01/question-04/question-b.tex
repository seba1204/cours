Pour $t\in\mathbb{R}$, posons :
\[
  f(t) =
  \left(
  \begin{array}{c}
      t                        \\
      t^{2} - 2\sqrt{2}\beta t \\
    \end{array}
  \right)
  \Rightarrow
  f'(t) =
  \left(
  \begin{array}{c}
      1                     \\
      2(t - 2\sqrt{2}\beta) \\
    \end{array}
  \right)
\]
Définissons l'angle de relèvement $\alpha$ : \[\forall t\in\mathbb{R}, \alpha(t) = \arctan{(\frac{z'(t)}{y'(t)})}\]
\[\iff\forall t\in\mathbb{R}, \alpha(t) = \arctan{(2(t-\sqrt{2}\beta))}\]
Alors :
\[
  \forall t\in\mathbb{R},\gamma(t) = \frac{\alpha'(t)}{\left\|f'(t)\right\|}
\]
De plus, $\alpha$ est dérivable sur $\mathbb{R}$ :
\[\forall t\in\mathbb{R}, \alpha'(t) = \frac{2}{1+4(t-\sqrt{2}\beta)^{2}}\]
Donc :
\[\forall t\in\mathbb{R}, \gamma(t) = \frac{2}{1+4(t-\sqrt{2}\beta)^{2}}\times \frac{1}{\sqrt{1+4(t-\sqrt{2}\beta)^{2}}}\]
Ainsi, $\gamma$ est maximal pour $(1+4(t-\sqrt{2}\beta)^{2})^{3/2}$ minimal i.e. $t=\sqrt{2}\beta$
