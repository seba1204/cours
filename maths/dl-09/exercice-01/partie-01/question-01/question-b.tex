Soit $x \in D$.
D'après l'ennoncé, on a :
\[
  \varphi(x) = \frac{1}{x}
  -
  \underbrace{\sum\limits_{n=1}^{+\infty} \left(\frac{1}{n-x}\right)}_{A(x)}
  +
  \underbrace{\sum\limits_{n=1}^{+\infty} \left(\frac{1}{n+x}\right)}_{B(x)}
\]
Ces séries $A(x)$ et $B(x)$ sont divergentes, passons donc par des somme partielles puis faisons tendre la borne $N$ vers + $\infty$. \smiley\\
Soit $N \in \mathbb{N}.$
Posons $\varphi_N(x) = \frac{1}{x}
  -
  \sum\limits_{n=1}^{+\infty} \left(\frac{1}{n-x}\right)
  +
  \sum\limits_{n=1}^{+\infty} \left(\frac{1}{n+x}\right)$.
Calculons $ \varphi_N(x + 1)$ :
\[
  \begin{array}{crcl}
         & \varphi_N(x + 1) & = & \frac{1}{x+1} - \sum\limits_{n=1}^{N} \left(\frac{1}{n-1-x}\right) + \sum\limits_{n=1}^{N} \left(\frac{1}{n+1+x}\right) \\\\
    \iff & \varphi_N(x + 1) & = & \frac{1}{x+1}
    -
    \sum\limits_{n=0}^{N-1} \left(\frac{1}{n-x}\right)
    +
    \sum\limits_{n=2}^{N+1} \left(\frac{1}{n+x}\right)
    \\\\
    \iff & \varphi_N(x + 1) & = & \frac{1}{x+1}
    -
    \left(\sum\limits_{n=1}^{N-1} \left(\frac{1}{n-x}\right) + \frac{1}{-x} - \frac{1}{N-x}\right)
    +
    \left(\sum\limits_{n=1}^{N+1} \left(\frac{1}{n+x}\right) - \frac{1}{1+x} + \frac{1}{N+1+x}\right)
    \\\\
    \iff & \varphi_N(x + 1) & = & \frac{1}{x}
    -
    \sum\limits_{n=1}^{N} \left(\frac{1}{n-x} - \frac{1}{n+x}\right) + \underbrace{\frac{1}{N+1+x} + \frac{1}{N-x}}_{\epsilon(x)}
    \\\\
    \iff & \varphi_N(x + 1) & = & \varphi_N(x) + \epsilon(x)
  \end{array}
\]
Or
\[
  \begin{array}{crcl}
                & \underset{N\rightarrow+\infty}{lim}\epsilon(x) & = & 0          \\ \\
    \Rightarrow & \varphi(x+1)                                   & = & \varphi(x) \\
  \end{array}
\]
Ainsi :
\begin{result}
  $\varphi$ est $1$-préiodique.
\end{result}

\textcolor{Blue}
{
  \textit
  {
    \ul{Remarques :}
    \begin{itemize}
      \item j'ai noté "1-périodique" comme on note "$\pi$-périodique" $\rightarrow$ vous m'avez déjà répondu en cours.
      \item j'ai utilisé les sommes partielles parce que vous nous l'avez dit en cours, je n'avais pas vu de problème à utiliser des séries qui divergent...
    \end{itemize}
  }
}
