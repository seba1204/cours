Je suppose que ce n'est pas $\sum\limits_{\textcolor{Red}{k}=1}^{\infty}\frac{1}{n^2}$ mais $\sum\limits_{\textcolor{Red}{n}=1}^{\infty}\frac{1}{n^2}$.\\
On a :
\[
  \begin{array}{lrcl}
                & \sum\limits_{n=1}^{\infty}\frac{1}{n^2-x^2}                               & =                  & \frac{1}{2x}(2-\varphi)                                       \\
                &                                                                           & =                  & \frac{1}{2x}(2-\psi)                                          \\
    \text{Or}   & \frac{1}{2x}(2-\psi)                                                      & \underset{0}{\sim} & \frac{1}{2x}\left(2-\frac{1}{x}+\frac{\pi^2}{3}x +o(x)\right) \\
                &                                                                           & \underset{0}{\sim} & \frac{\pi^2}{6}+o(1)                                          \\
    \text{Donc} & \underset{x\rightarrow 0}{lim}\sum\limits_{n=1}^{\infty}\frac{1}{n^2-x^2} & =                  & \frac{\pi^2}{6}                                               \\
    \Rightarrow & \sum\limits_{n=1}^{\infty}\frac{1}{n^2}                                   & =                  & \frac{\pi^2}{6}                                               \\
  \end{array}
\]
D'où
\begin{result}
  $\sum\limits_{n=1}^{\infty}\frac{1}{n^2}=\frac{\pi^2}{6}$
\end{result}
\textcolor{Blue}{\textit{\ul{Remarques :}
    \begin{itemize}
      \item je ne pense pas avoir bien rédiger...
      \item il me semble qu'en cours on avait vu la fonction zeta de Riemann, et : \[\sum\limits_{n=1}^{\infty}\frac{1}{n^2} = \zeta(2) = \frac{\pi^2}{6}\]
    \end{itemize}
  }}
