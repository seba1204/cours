On a donc ; $T_n : F_n \rightarrow F_n$. \textit{Les fonctions sont donc maintenant uniquement des polynômes}\\
Montrons que $T_n$ est diagonalisable.
Notons d'abord que $\mathcal{B}_n$ est une base car une famille de degré échelonnée donc libre et génératrice de $F_n$ car $\mathcal{B}_n = Vect\{F_n\}$.
Ecrivons la matrice associée à $T_n$ dans la base $\mathcal{B}_n$. On a:

\[
  \begin{array}{rcl}
    T_{n}(e_j) & = & x \mapsto \frac{x^j}{2^j} + \frac{(x+1)^j}{2^j}  \\
               & = & x \mapsto \frac{x^j}{2^j} + \frac{1}{2^j} \times
    \sum\limits_{k=0}^{j}
    \left(
    \begin{array}{c}
        j \\
        k
      \end{array}
    \right) x^k
    \\
  \end{array}
\]
Donc :
\[
  \underset{\mathcal{B}_n, \mathcal{B}_n}{mat(T_n)} = A =
  \begin{blockarray}{ccccccc}
    T_n(e_1) & T_n(e_2) & \dots &T_n(e_j) & \dots & T_n(e_n) \\
    \begin{block}{(cccccc)c}
      2       & a_{12}  & \dots   & a_{1j}  & \dots   & a_{1n}  & e_1     \\
      0       & 1       & ~       & ~       & ~       & \vdots  & e_2     \\
      \vdots  & \ddots  & \ddots  & ~       & ~       & \vdots  & \vdots  \\
      \vdots  & ~       & \ddots  & 2^{1-i} & ~       & a_{in}  & e_i     \\
      \vdots  & ~       & ~       & \ddots  & \ddots  & \vdots  & \vdots  \\
      0       & \dots   & \dots   & \dots   & 0       & 2^{1-n} &  e_n    \\
    \end{block}
  \end{blockarray}
\] On remarque donc que $A$ est triangulaire supérieure, et donc, par propriété son déterminant est le produit des coefficients de la diagnoale. On en déduit son polynôme caractéristique :
\[
  \chi(x) = \prod_{k=0}^{n} \left(x-\frac{1}{2^{k-1}}\right)
\]
$\chi$ est donc scindé à racines simples dans $\mathbb{R}$, par théorème, A est diagonalisable, d'où :
\begin{result}
  $T_n$ est diagonalisable.
\end{result}
