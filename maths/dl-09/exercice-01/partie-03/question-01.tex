$cotan(x) = \frac{cos(x)}{sin(x)}$\\
Soit $x\in\mathbb{R}$.\\
D'après le théorème de la bijection écrit dans mon cours de PTSI :
\\Si :
\begin{itemize}
  \item $f$ est strictement monotone sur $I$
  \item $f$ et continue sur $I$
\end{itemize}
Alors :
\begin{itemize}
  \item $f$ réalise une bijection de $I$ sur $f(I)$
  \item $f$  et $f^{-1}$ ont la même monotonie
  \item $f^{-1}$ est continue
\end{itemize}

\begin{dinglist}{111}
  \item \ul{Continuit$\acute{e}$ :}
  $cotan$ est continue sur $]0, \pi[$ car $sin$ s'annule en $0$ et $\pi$.
        \item \ul{Monotonie :} Calculons la dérivée.\\
      $cotan$ est donc dérivable sur $]0, \pi[$ et : \\
      $\forall x \in ]0, \pi[, cotan'(x) = \frac{-sin^2(x)-cos^2(x)}{sin^2(x)}=-\frac{1}{sin^2(x)}$
        Ainsi, $cotan$ est strictement décroissante sur $]0, \pi[$.
\end{dinglist}
Par théorème :
\begin{result}
  $f$ réalise une bijection de $I = ]0, \pi[$ sur $f(I) = \mathbb{R}$
\end{result}
