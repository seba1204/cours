Soit $y \in \mathbb{R}$. On cherche $x\in I$ tel que $y = cotan(x)$
\[
  \begin{array}{rl}
    \iff & y = \frac{cos(x)}{sin(x)}          \\
    \iff & y  = \frac{1}{tan(x)}              \\
    \iff & x = arctan\left(\frac{1}{y}\right) \\
    \iff &
  \end{array}
\]
Or, on a la relation : $\forall x \in\mathbb{R}, arctan(x)+arctan\left(\frac{1}{x}\right) = \frac{\pi}{2}$. D'où
\[
  \begin{array}{rl}
    \iff & x = \frac{\pi}{2} - arctan(x)
  \end{array}
\]
Donc :
\begin{result}
  $\forall y\in \mathbb{R}, arccotan(y) = \frac{\pi}{2} - arctan(y)$
\end{result}
