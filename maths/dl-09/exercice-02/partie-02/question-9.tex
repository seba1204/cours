$G_Z(t)$ est une série entière, par propriété elle est de classe $\mathcal{C}^{+\infty}$ sur $]-a, a[$.\\
        Or $1\in]-a, a[$, donc $G_Z'(1)$ et $G_Z''(1)$ existent.\\
Par propriété :
\[
  \left\{
  \begin{array}{rcl}
    E(Z) & = & G_Z'(1)                        \\
    V(Z) & = & G_Z'(1) + G_Z''(1) - G_Z'(1)^2 \\
  \end{array}
  \right.
\]
De plus, en reprenant la formule : $G_Z(t)(1-pt-pqt^2) = q^2t^2$, on trouve :
\[
  \begin{array}{rrcl}
                & G_Z'(t) & = & \frac{2tq^2(1-pt-pqt^2) - t^2q^2(-p-2pqt)}{(1-pt-pqt^2)^2} \\ \\
    \Rightarrow & G_Z'(1) & = & \frac{2q^2(1-p-pq) - q^2(-p-2pq)}{(1-p-pq)^2}              \\ \\
    \Rightarrow & G_Z'(1) & = & \frac{2q^2(q^2) - q^2(-p-2pq)}{(q^2)^2}                    \\ \\
    \Rightarrow & G_Z'(1) & = & \frac{2q^2 +p+2pq}{q^2}                                    \\ \\
    \Rightarrow & G_Z'(1) & = & 2 + \frac{1-q}{q^2}   + 2\frac{1-q}{q}                     \\ \\
    \Rightarrow & G_Z'(1) & = & \frac{1}{q^2}   + \frac{1}{q}                              \\ \\
  \end{array}
\]
D'où
\begin{result}
  $E(Z) = \frac{1}{q^2}   + \frac{1}{q}$
\end{result}
