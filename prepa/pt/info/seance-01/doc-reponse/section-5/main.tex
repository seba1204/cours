\section{Exercice 2 : Nombres premiers}
\begin{enumerate}
  \item
        \q{En utilisant}
        \inlineCode{remove}
        \q{donner la liste des 168 nombres premiers de 1 à 1 000 en utilisant le crible d'Eratosthène. Faire moins de 1 500 tours de boucle. De même trouver les 1 229 nombres premiers de 1 à 10 000 en faisant moins de 20 000 tours de boucle.}
\end{enumerate}

\codeFromFile{section-5/exercice_2.py}

Résultats :
\begin{code}
  ========= Test pour n = 1000 =========
  Le programme fait 831 tours (max : 1500 tours).
  Temps d execution : 0.008s
  Nombre de resultats : 168
  Resultats :[2, 3, 5, 7, '...', 991, 997]

  ========= Test pour n = 10000 =========
  Le programme fait 8770 tours (max : 20000 tours).
  Temps d execution : 0.595s
  Nombre de resultats : 1229
  Resultats :[2, 3, 5, 7, '...', 9967, 9973]
\end{code}
