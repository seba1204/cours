\section{Vérification du parenthésage d'une expression}
\q{On considère des expressions algébriques données sous forme de chaîne :}
\begin{center}
  \il{c1 = '((3*5-7)+1'}\\
  \il{c2 = '((3*5)-7)+1'}\\
  \il{c3 = '(3*5)-(7)+1)'}\\
  \il{c4 = '(3*(5-7)+(1'}
\end{center}
\q{On souhaite faire une analyse synthaxique pour vérifier qu'à chaque parenthèse ouvrante correspond une unique parentèse fermante.}
\begin{enumerate}
  \item \q{Proposer un algorithme utilisant une pile et réalisant ce test.}\\
        On crée un pile vide.\\
        A chaque parenthèse ouvrante, on empile $1$.\\
        A chaque parenthèse fermante, on dépile $1$.\\
        A la fin, si la pile n'est pas vide, c'est qu'il y a une erreur dans l'expression algébrique.
  \item \q{Ecrire une fonction }\il{test_parentheses(c)}\q{ qui implémente cet algorithme.}
        \codeFromFile{section-02/q1.py}
\end{enumerate}
