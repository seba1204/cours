\section{Numéro d'une date dans l'année}
\begin{enumerate}[(a)]
  \item
        \q{Une année est bissextile si elle est divisible par 400 ou si elle est divisible par 4 mais pas par 100.
          Ecrire une fonction }\il{bissextile}\q{ qui renvoie un booléen suivant qu'une année }\il{a}\q{ est bissextile
          ou non.}

        \bigskip

        \codeFromFile{section-02/qa.py}
  \item
        \q{Donner la liste de toutes les années bissextiles de 1900 à 2020 et vérifier que la somme de ces années vaut
          58 860.}

        \bigskip

        \codeFromFile{section-02/qb.py}

        \bigskip

        En exécutant \il{print(sumYears(1900, 2020))} on obtient :
        \[
          \left[
            \begin{array}{cccccc}
              1904 & 1924 & 1944 & 1964 & 1984 & 2004 \\
              1908 & 1928 & 1948 & 1968 & 1988 & 2008 \\
              1912 & 1932 & 1952 & 1972 & 1992 & 2012 \\
              1916 & 1936 & 1956 & 1976 & 1996 & 2016 \\
              1920 & 1940 & 1960 & 1980 & 2000 & 2020 \\
            \end{array}
            \right]
        \]
        \[
          \il{sum(A) = 58 860}
        \]

  \item
        \q{On donne les listes des mois courts et des mois longs :}
        \[
          \begin{array}{l}
            \il{ML = [1, 3, 5, 7, 8, 10, 12]} \\\\
            \il{MC = [2, 4, 5, 9, 11]}        \\
          \end{array}
        \]
        \q{Ecrire une fonction }\il{jour_annee(date)}\q{ qui renvoie le numéro de l'année en cours.}
        \bigskip

        \codeFromFile{section-02/qc.py}

  \item
        Le jeudi 12 décembre 2019 sera un jour de pleine lune. En regardant un vieux calendrier des postes, Sophie a
        constaté qu'une pleine lune a eu lieu le 15 janvier 1900. Comme dans sa famille on collection les calendriers
        des postes depuis cette époque, Sophie a pu dénombrer 1484 pleines lunes dans les 120 calendriers de 1900 à 2019.
        \q{Trouver avec Sophie la révolution synodique de a lune.}

        \codeFromFile{section-02/qd.py}

        \bigskip

        En exécutant \il{print(synodique(1900, 2019, 1484))}, on trouve :
        \begin{result}
          La période synodique de la lune est 29.534 jours.
        \end{result}

  \item
        \q{La célèbre nuit du 4 au 5 août 1789 était-elle une nuit de pleine lune ?}

        \codeFromFile{section-02/qe.py}

        \bigskip

        De même, en exécutant \il{print(isFullMoonMdr([4, 8, 1789]))}, on trouve \il{False} :
        \begin{result}
          La fameuse nuit du 4 au 5 août 1789 n'était pas une nuit de pleine lune.
        \end{result}
\end{enumerate}


