Passons par les sommes partielles.
Soit $N\in\mathbb{N}$

\[
  \begin{array}{rcl}
    \varphi_N\left(\frac{x}{2}\right) & = & \frac{2}{x}
    -
    \sum\limits_{n=1}^{N} \left(\frac{2}{2n-x}\right)
    +
    \sum\limits_{n=1}^{N} \left(\frac{2}{2n+x}\right)
    \\\\
                                      & = & 2 \left[
      \frac{1}{x}
      -
      \sum\limits_{n=2}^{2N} \left(\frac{1}{n-x}\right)
      +
      \sum\limits_{n=2}^{2N} \left(\frac{1}{n+x}\right)
      \right]
    \\\\
  \end{array}
\]
Or,
\[
  \left\{
  \begin{array}{rcl}
    \sum\limits_{n=2}^{2N} \left(\frac{1}{n-x}\right) & = & -\frac{1}{1-x} + \sum\limits_{n=1}^{N} \left(\frac{1}{n-x}\right) + \sum\limits_{n=N+1}^{2N} \left(\frac{1}{n-x}\right) \\
    \sum\limits_{n=2}^{2N} \left(\frac{1}{n+x}\right) & = & -\frac{1}{1+x} + \sum\limits_{n=1}^{N} \left(\frac{1}{n+x}\right) + \sum\limits_{n=N+1}^{2N} \left(\frac{1}{n+x}\right) \\
  \end{array}
  \right.
\]
De même :

\[
  \begin{array}{rcl}
    \varphi_N\left(\frac{x+1}{2}\right) & = & 2 \left[
      \frac{1}{x+1}
      -
      \sum\limits_{n=2}^{2N} \left(\frac{1}{n-x-1}\right)
      +
      \sum\limits_{n=2}^{2N} \left(\frac{1}{n+x+1}\right)
      \right]
    \\\\
                                        & = & 2 \left[
      \frac{1}{x+1}
      -
      \sum\limits_{n=1}^{2N-1} \left(\frac{1}{n-x}\right)
      +
      \sum\limits_{n=3}^{2N+1} \left(\frac{1}{n+x}\right)
      \right]
  \end{array}
\]
Et,
\[
  \left\{
  \begin{array}{rcl}
    \sum\limits_{n=1}^{2N-1} \left(\frac{1}{n-x}\right) & = & -\frac{1}{1-x} - \frac{1}{2N-x} + \sum\limits_{n=1}^{N} \left(\frac{1}{n-x}\right) + \sum\limits_{n=N+1}^{2N} \left(\frac{1}{n-x}\right)                \\
    \sum\limits_{n=3}^{2N+1} \left(\frac{1}{n+x}\right) & = & -\frac{1}{1+x} -\frac{1}{2+x} + \frac{1}{2N-x} + \sum\limits_{n=1}^{N} \left(\frac{1}{n+x}\right) + \sum\limits_{n=N+1}^{2N} \left(\frac{1}{n+x}\right) \\
  \end{array}
  \right.
\]

En sommant tous les termes, on trouve :
\[
  \begin{array}{rcl}
    \varphi_N\left(\frac{x}{2}\right) + \varphi_N\left(\frac{x+1}{2}\right) & = & 2 \varphi_N(x) + \frac{2}{2N-x}
  \end{array}
\]
Donc, lorsque $N \rightarrow +\infty$ :
\begin{result}
  $
    \begin{array}{rcl}
      \varphi\left(\frac{x}{2}\right) + \varphi\left(\frac{x+1}{2}\right) & = & 2 \varphi(x)
    \end{array}
  $
\end{result}
