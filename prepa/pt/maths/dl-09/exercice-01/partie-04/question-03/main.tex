Pour un peu plus de clareté, notons $\Psi = \varphi-\psi.$\\
$\Psi \in \mathcal{C}([0, 1])\Rightarrow\Psi\in E$. On peut donc appliquer $T$ à $\Psi$.\\

Notons de plus que, grâce aux questions III 4. et IV 1., on peut écrire :
\[
  \begin{array}{rl}
                & T(\Psi) = 2\Psi                             \\
    \Rightarrow & \Psi\in Ker(T-2Id_E)                        \\
    \Rightarrow & \forall x \in [0, 1], \Psi(x) = \Psi(0) = 0 \\
    \Rightarrow & \varphi=\psi                                \\
  \end{array}
\]
On a donc montré que : 
\begin{result}
  $\varphi=\psi $
\end{result}
