Redémontrons la formule de la série génératrice d'une loi géométrique. On se souvient de bien faire attention aux indices, le seul piège.\\
\[
  \begin{array}{rrcl}
                & G_y(t) & = & \sum\limits_{n=1}^{+\infty}P(Y=n)t^n               \\\\
    \Rightarrow & G_y(t) & = & \sum\limits_{n=1}^{+\infty}(1-q)^{n-1}qt^n         \\\\
    \Rightarrow & G_y(t) & = & \frac{q}{p}\sum\limits_{n=1}^{+\infty}(pt)^{n}     \\\\
    \Rightarrow & G_y(t) & = & \frac{q}{p}pt\sum\limits_{n=1}^{+\infty}(pt)^{n-1} \\\\
    \Rightarrow & G_y(t) & = & qt\sum\limits_{n=0}^{+\infty}(pt)^{n}              \\\\
    \Rightarrow & G_y(t) & = & qt\frac{1}{1-pt}                                   \\\\
    \Rightarrow & G_y(t) & = & \frac{qt}{1-pt}                                    \\\\
  \end{array}
\]
De plus : $G_y(t) = \frac{q}{p}\sum\limits_{n=1}^{+\infty}(pt)^{n}$ converge pour $|pt|<1\Rightarrow|t|<\frac{1}{p}$. Donc le rayon de convergence est $R_y = \frac{1}{p}$ avec $p\in]0, 1[$.

Ainsi :
\begin{result}
  $
    \left\{
    \begin{array}{rcl}
      R_y                             & = & \frac{1}{p} > 1 \\
      \forall t\in]-R_y, R_y[, G_y(t) & = & \frac{qt}{1-pt} \\
    \end{array}
    \right.
  $
\end{result}
