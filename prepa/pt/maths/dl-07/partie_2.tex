\section{Crête et talweg sur $\Sigma$}
\begin{enumerate}
  \item On reprend l'équation paramétrique de $\Sigma$. Soit $\theta\in[0,\pi[$ :
        \begin{result}
          $
            \Gamma_\theta :
            \left\{
            \begin{array}{rcl}
              x(r) & = & r \times \cos{(\theta)}                       \\
              y(r) & = & r \times \sin{(\theta)}                       \\
              z(r) & = & 3r^{2} \sin{(\theta)}\sin{(\theta - \varphi)} \\
            \end{array}
            \right.
          $
        \end{result}
  \item
        On a deux vecteurs directeurs de $P_\theta$ :
        \[
          \overrightarrow{\delta_\theta} =
          \left(
          \begin{array}{c}
              \cos{(\theta)} \\
              \sin{(\theta)} \\
              0              \\
            \end{array}
          \right) et \overrightarrow{k} =
          \left(
          \begin{array}{c}
              0 \\
              0 \\
              1 \\
            \end{array}
          \right)
        \] On a alors : $\Gamma_\theta : r(\overrightarrow{\delta_\theta} + 3r\sin{(\theta)}\sin{(\theta - \varphi)}\overrightarrow{k})$. Donc $\Gamma_\theta \subset P_\theta$.

  \item Dans le repère $(O, \overrightarrow{\delta_\theta}, overrightarrow{k})$ :
        \[
          y = x^{2} \times sin(\theta)sin(\theta - \varphi)
        \]
        Ainsi $\Gamma_\theta$ est une parabole si $sin(\theta)sin(\theta - \varphi) \neq 0$.
        \begin{result}
          $\Gamma_\theta$ est une parabole si $\theta\neq 0$ et $\theta\neq \varphi$.
        \end{result}
  \item \begin{enumerate}
          \item On a, d'après la partie 1 :
                \[
                  \overrightarrow{n} =
                  \iff r^{2}
                  \left(
                  \begin{array}{c}
                      3\sin{(\theta_0)}\left[ \sin{(2\theta_0 - \varphi)} - 2\sin{(\theta_0 - \varphi)}\cos{(\theta_0)}\right]       \\
                      3\left[ 2\sin^{2}{(\theta_0)}\sin{(\theta_0 - \varphi)} + \cos{(\theta_0)} \sin{(2\theta_0 - \varphi)} \right] \\
                      1                                                                                                              \\
                    \end{array}
                  \right)
                \]
          \item Calculer l'intersection des deux : $P\theta$ et $\mathcal{N}_\theta(r)$
          \item $2\theta = arctan(2\sqrt{2}) = \varphi$ ou $2\theta = \varphi + \frac{\pi}{2}$
                \begin{result}
                  $\alpha = \frac{\varphi}{2}$ et $\beta = \frac{\varphi}{2} + \frac{\pi}{4}$
                \end{result}
        \end{enumerate}
  \item \textit{Pas de question 5...}
  \item \begin{enumerate}
          \item On a $x = 0$ d'où :
                \begin{result}
                  $
                    \left\{
                    \begin{array}{rcl}
                      y & = & rsin(\theta)                     \\
                      z & = & 3rsin(\theta)sin(\theta-\varphi) \\
                    \end{array}
                    \right.
                  $
                \end{result}
                De plus on a $f(y,z) = f(-y,z)$ donc
                \begin{result}
                  L'axe $(O, z)$ est un axe de symétrie.
                \end{result}
          \item On calcule la $f'$ en $0$ et on trouve :
                \begin{result}
                  $
                    \left\{
                    \begin{array}{rcl}
                      z & = & -y \\
                      z & = & y  \\
                    \end{array}
                    \right.
                  $
                \end{result}
        \end{enumerate}
  \item \begin{enumerate}
          \item Il faut trouver pour quels $\theta,  z'(\theta) = 0$  sur $[0, 2\pi[$
        \end{enumerate}
\end{enumerate}
