$Q(t)$ est un polynôme de degré $2$, il peut donc se mettre sous la forme :
$Q(t) = \alpha (t-r_1)(t-r_2)$
avec $r_1$ et $r_2$ les racines de $Q$ et $\alpha$ le coefficient dominant.
D'où :
\begin{result}
  $Q(t) = -pt(t-a)(t-b)$
\end{result}
