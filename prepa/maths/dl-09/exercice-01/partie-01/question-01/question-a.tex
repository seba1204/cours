$D = \mathbb{R} - \mathbb{Z}$ Pour $x \in D$, on a :
\[
  \varphi(x) = \frac{1}{x} - \sum_{n=1}^{+\infty} \frac{2x}{n^2-x^2} =  \frac{1}{x} - \sum_{n=1}^{+\infty} u_n(x)
\]
Soit $x$ un réel.\\
Pour que $\varphi(x)$ soit définie sur $D$, il faut que $\forall x \in D$, $\varphi(x)$ converge. Décomposons $\varphi(x)$ en deux : \\
\[
  \varphi(-x) = \underbrace{\frac{1}{x}}_{A(x)} - \underbrace{\sum\limits_{n=1}^{+\infty} \frac{2x}{n^2-x^2}}_{B(x)}
\]
\begin{dinglist}{111}
  \item $A(x)$ existe pour $x\neq 0$.
  \item $B(x)$ n'existe pas si $x = n\in\mathbb{N}$. De plus, pour $x\in D$, on a :
  \[\frac{2x}{n^2-x^2}\underset{n\rightarrow+\infty}{\sim}\frac{1}{n^2}\]
  Ainsi, par le théorème de compraison des séries à termes positifs, $B(x)$ a même nature que $\sum\limits_{n=0}^{+\infty} \frac{1}{n^2}$.
  Or, la série de référence de Riemann $\sum\limits_{n=0}^{+\infty} \frac{1}{n^2}$ converge. Alors $B(x)$ converge.
\end{dinglist}
D'où, $\varphi$ bien définie sur $D$.

De plus, pour $x \in D$ :
\[
  \begin{array}{crcl}
         & \varphi(-x) & = & \frac{1}{-x} - \sum\limits_{n=1}^{+\infty} \frac{2(-x)}{n^2-x^2}           \\\\
    \iff & \varphi(-x) & = & -\frac{1}{x} + \sum\limits_{n=1}^{+\infty} \frac{2x}{n^2-x^2}              \\\\
    \iff & \varphi(-x) & = & -\left(\frac{1}{x} - \sum\limits_{n=1}^{+\infty} \frac{2x}{n^2-x^2}\right) \\\\
    \iff & \varphi(-x) & = & -\varphi(x)                                                                \\\\
  \end{array}
\]
Ainsi :
\begin{result}
  $\varphi$ est bien définie sur $D$ et $\varphi$ est \textbf{impaire}.
\end{result}
