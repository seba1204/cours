Je suppose qu'ici c'est le piège d'inverser limite et somme infinie :
\[
  \underset{x\rightarrow0}{lim}\left( \sum\limits_{n=0}^{+\infty} \frac{2x}{n^2-x^2}\right)\underset{\text{pas forcément}}{=}0
\]
\textit{Ha oui mais non} car ici $g(0)$ existe d'après la question 2.(b). Donc on peut le calculer directement, sans faire de limite ?\\
$g(0) = 0$ et
\[
  \begin{array}{rl}
                & \varphi(x) = \frac{1}{x} + \frac{2x}{1-x^2} - g(x)     \\\\
    \Rightarrow & \varphi(x) \underset{x\rightarrow 0}{\sim} \frac{1}{x}
  \end{array}
\]
De plus, comme $\varphi$ est 1-périodique, $\varphi$ aura le même équivalent en $0$ qu'en $0+1$

D'où
\begin{result}
  $
    \left\{
    \begin{array}{c}
      \varphi(x) \underset{x\rightarrow 0}{\sim} \frac{1}{x} \\\\
      \varphi(x) \underset{x\rightarrow 1}{\sim} \frac{1}{x}
    \end{array}
    \right.
  $
\end{result}
