Pour $n \in \mathbb{N}$ on définit "$P(n) : f\left(\frac{x_0}{2^n}\right)=m$".
\begin{dinglist}{111}
  \item \ul{Initialisation :} pour $n=0$\\
  $f(x_0)=m$ : OK d'après la question 3.(b)
  \item \ul{Heredite :}
  pour $n\in\mathbb{N}$ fixé, supposons $P(n)$ vraie, montrons que $P(n+1)$ l'est aussi.
  On a, d'après le raisonnement précédent :
  \[
    f\left(\frac{x_0}{2^n}\right)=m \Rightarrow f\left(\frac{x_0}{2^{n+1}}\right)=m
  \]
  D'où $P(n+1)$ vraie.
  \item \ul{Conclusion :}
  On a montré l'initialisation et l'hérédité de $P$. Par le principe de démonstration par récurrence, on a montré :
  \begin{result}
    $\forall n \in\mathbb{N}, f\left(\frac{x_0}{2^{n}}\right) = m$
  \end{result}
\end{dinglist}

De plus :
\[
  \begin{array}{rl}
                & \underset{n\rightarrow+\infty}{lim} f\left(\frac{x_0}{2^{n}}\right) = m \\\\
    \Rightarrow & f\left(0\right) = m                                                     \\
  \end{array}
\]
Donc
\begin{result}
  $f\left(0\right) = m  $
\end{result}
