$\Gamma$ est une intégrale à paramètre, appliquons le théorème du cours :\\
On a :
\begin{dinglist}{111}
  \item \ul{Regularite selon $x\in\mathbb{R}$ :}\\
  $
    \forall t\in ]0, +\infty[, h(\blacksquare, t) :
    \begin{array}{l}
      \mathbb{R}\rightarrow\mathbb{R} \\
      x\mapsto h(x, t)
    \end{array}
  $
  continue sur $\mathbb{R}$
  \item \ul{Regularite selon $t\in]0, +\infty[$ :}\\
    $
    \forall x\in \mathbb{R}, h(x, \blacksquare) :
    \begin{array}{l}
      ]0, +\infty[\rightarrow\mathbb{R} \\
      t\mapsto h(x, t)
    \end{array}
    $
      continue sur $ ]0, +\infty[$ car $\forall t>0, (e^t-1)>0$
    \item \ul{Domination :}\\
    D'après la question précédente :
    \[\exists M\in\mathbb{R} / \forall (x, t)\in \mathbb{R}\times]0, +\infty[, |h(x, t)| \leq M |x| \underbrace{\frac{t}{e^t-1}}_{\varphi(t)}\]
  Or,
  \[
    \left\{
    \begin{array}{rcl}
      e^t-1 & \underset{+\infty}{\sim} & e^t        \\
      t     & \underset{+\infty}{=}    & o(e^{t/2}) \\
    \end{array}
    \right.
  \]
  \[
    \begin{array}{rrcl}
      \Rightarrow & \frac{t}{e^t} & \underset{+\infty}{=} & o(e^{-t/2}) \\
      \Rightarrow & \varphi(t)    & \underset{+\infty}{=} & o(e^{-t/2}) \\
    \end{array}
  \]
  Or, $e^{-t/2}$ est intégrale sur $[0, +\infty$ en tant qu'intégrale de référence. Par le théorème de comparaison des fonctions positives, $\varphi(t)$ est intégrale sur $[0, +\infty[$.
\end{dinglist}
Par théorème :
\begin{result}
  $\Gamma$ continue sur $\mathbb{R}$
\end{result}
