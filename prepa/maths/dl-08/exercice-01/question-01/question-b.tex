Soit $t \in \mathbb{R}$.
$D_t = (P_tQ_t) = P_t + Vect \underbrace{\left\{\overrightarrow{P_tQ_t}\right\}}_{\overrightarrow{u_t}}$. \\
On a : $P_t, \overrightarrow{u_t}$ de classe $\mathcal{C}^1$ sur $\mathbb{R}$ et : $\exists t\in\mathbb{R}, det(\overrightarrow{u_t}, \overrightarrow{u_t}')\neq0$. Donc, par propriété, l'enveloppe des droites $D_t$, nommée $\Gamma$, existe.\\
Paramétrons $\Gamma$ par : $\overrightarrow{f}(t) = \overrightarrow{OC}(t) $.\\
Pour $t \in\mathbb{R}$, on pose $\lambda(t)\in\mathbb{R}$ l'inconnue.\\
Par définition de l'enveloppe, $\Gamma$ doit respecter deux conditions.
\begin{enumerate}[label=\arabic*. ]
  \item Intersection de $\Gamma$ avec $D_t$ :
        \[
          \overrightarrow{f}(t) = P_t + \lambda(t)\overrightarrow{u_t}
        \]
  \item $D_t$ tangente à $\Gamma$ en $C$ :
        \[
          \begin{array}{rl}
            \text{i.e.} & det(\overrightarrow{f'}(t), \overrightarrow{u_t}) = 0                                                    \\
            \iff        & det(
            P_t' + \lambda'(t)\overrightarrow{u_t} + \lambda(t)\overrightarrow{u_t}'
            , \overrightarrow{u_t}
            ) = 0                                                                                                                  \\
            \iff        & det(
            P_t' + \lambda(t)\overrightarrow{u_t}'
            , \overrightarrow{u_t}
            ) + \lambda'(t)\underbrace{det(
              \overrightarrow{u_t}
              , \overrightarrow{u_t}}_{0}
            ) = 0                                                                                                                  \\
            \iff        & det(
            P_t', \overrightarrow{u_t})
            + \lambda(t)det(\overrightarrow{u_t}', \overrightarrow{u_t}) = 0                                                       \\
            \iff        & \lambda(t) = \frac{det(P_t',  \overrightarrow{u_t})}{det( \overrightarrow{u_t},  \overrightarrow{u_t}')} \\
          \end{array}
        \]
        On remplace :
        \[
          \begin{array}{rcl}
            P_t'                  & = &
            \left(
            \begin{array}{c}
                sin(t) \\cos(t)
              \end{array}
            \right)                     \\
            \overrightarrow{u_t}  & = &
            \left(
            \begin{array}{c}
                cos(2t) + cos(t) \\
                sin(2t) - sin(t) \\
              \end{array}
            \right)                     \\
            \overrightarrow{u_t}' & = &
            \left(
            \begin{array}{c}
                -2sin(2t) - sin(t) \\
                2cos(2t) - cos(t)  \\
              \end{array}
            \right)
          \end{array}
        \]
        D'où
        \[
          \forall t\in\mathbb{R},\lambda(t) = -\frac{cos(3t) + 1}{cos(3t) + 1} = (-1)
        \]
\end{enumerate}
En reprenant la condition 1. $\overrightarrow{f}(t) = P_t + \lambda(t)\overrightarrow{u_t}$ :
\[
  \overrightarrow{f}(t) =
  \left(
  \begin{array}{c}
      -cos(t) \\
      sin(t)
    \end{array}
  \right)
  +
  (-1)
  \times
  \left(
  \begin{array}{c}
      cos(2t) + cos(t) \\
      sin(2t) - sin(t) \\
    \end{array}
  \right)
\]
Ainsi, on obtient :
\[
  \overrightarrow{f}(t) =
  \left(
  \begin{array}{c}
      x(t) \\
      y(t)
    \end{array}
  \right) =
  \left(
  \begin{array}{c}
      -2cos(t)-cos(2t) \\
      2sin(t)-sin(2t)
    \end{array}
  \right)
\]
D'où :
\begin{result}
  $M_t\in\Gamma\iff\left\{
    \begin{array}{rcl}
      x(t) & = & -2cos(t)-cos(2t) \\
      y(t) & = & 2sin(t)-sin(2t)
    \end{array}
    \right.$
\end{result}

\begin{flushright}
  CQFD
\end{flushright}
