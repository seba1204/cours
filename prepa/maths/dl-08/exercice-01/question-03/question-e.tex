On conserve toujours la même fonction $\overrightarrow{f}$.\\
On a trouvé à la question 3. (d) que le vecteur tangent à $\Gamma_1$ en $I$ est :
$ \overrightarrow{u} =
  \left(
  \begin{array}{c}
      -\frac{3}{2} \\ -3\sqrt{3}
    \end{array}
  \right)
$
\[
  \begin{array}{cl}
      & I + Vect\left\{\overrightarrow{u}\right\} \\
    = & \left(
    \begin{array}{c}
        \frac{3}{2}         \\
        \frac{3\sqrt{3}}{2} \\
      \end{array}\right)
    + Vect\left\{\left(
    \begin{array}{c}
        \frac{3}{2}         \\
        \frac{3\sqrt{3}}{2} \\
      \end{array}\right)
    \right\}                                      \\
    = & \left(
    \begin{array}{c}
        \frac{3}{2}         \\
        \frac{3\sqrt{3}}{2} \\
      \end{array}\right)
    + \lambda\left(
    \begin{array}{c}
        \frac{3}{2}         \\
        \frac{3\sqrt{3}}{2} \\
      \end{array}\right)              \\
  \end{array}
\]
\[
  \iff
  \left\{
  \begin{array}{rcl}
    x & = & \frac{3}{2}(1-\lambda)                    \\
    y & = & 3\sqrt{3}\left(\frac{1}{2}-\lambda\right) \\
  \end{array}
  \right.
\]
\[
  \iff
  \left\{
  \begin{array}{rcl}
    \lambda & = & -\frac{3}{2}x-1                           \\
    y       & = & 3\sqrt{3}\left(\frac{1}{2}-\lambda\right) \\
  \end{array}
  \right.
\]
\[
  \begin{array}{lrcl}
    \Rightarrow & y & = & 3\sqrt{3}\left(\frac{1}{2}+\frac{3}{2}x+1\right) \\
    \Rightarrow & y & = & \frac{9\sqrt{3}}{2}\left(1+x\right)              \\
  \end{array}
\]
