Faison de même pour le point $I$, mais cette fois en dérivant.\\
On reprant la même fontion $\overrightarrow{f}$.\\
$\overrightarrow{f}$ est de classe $\mathcal{C}^\infty$ sur $\mathbb{R}$ en tant que composition de fonctions qui le sont.\\
\[
  \begin{array}{crcl}
                           & \overrightarrow{f}(t) & = &
    \left(
    \begin{array}{c}
        -2cos(t)-cos(2t) \\
        2sin(t)-sin(2t)  \\
      \end{array}
    \right)                                              \\\\
    \Rightarrow            &
    \overrightarrow{f}'(t) & =                     &
    \left(
    \begin{array}{c}
        2sin(t)+2sin(2t) \\
        2cos(t)-2cos(2t) \\
      \end{array}
    \right)                                              \\
  \end{array}
\]
Or, pour $t = \frac{2\pi}{3}$, $\overrightarrow{f}'(t)=(0, 0)$. On continue donc à dériver.
\[
  \overrightarrow{f}''(t)=
  \left(
  \begin{array}{c}
      2cos(t)+4cos(2t)  \\
      -2sin(t)+4sin(2t) \\
    \end{array}
  \right)
\]
Pour $t = \frac{2\pi}{3}$, $\overrightarrow{f}''(t)=\left(-\frac{3}{2}, -3\sqrt{3}\right)$. Voici notre première dérivée non nulle.
\[
  \overrightarrow{f}'''(t)=
  \left(
  \begin{array}{c}
      -2sin(t)-8sin(2t) \\
      -2cos(t)+8cos(2t) \\
    \end{array}
  \right)
\]
De même, pour $t = \frac{2\pi}{3}$, $\overrightarrow{f}'''(t)=\left(3\sqrt{3}, -3\right)$. Voici la seconde.
\\De plus, on a bien la famille
$\mathcal{F} = \left(
  \left(
    \begin{array}{c}
        -\frac{3}{2} \\ -3\sqrt{3}
      \end{array}
    \right),
  \left(
    \begin{array}{c}
        3\sqrt{3} \\ -3
      \end{array}
    \right)
  \right)$ libre
car $det(\mathcal{F}) = 4 \neq 0$.\\
Ainsi, ici aussi, on a "$(p, q)= (2, 3)$". Donc, par propiété :
\begin{result}
  $I$ est un point de \textbf{rebroussement de 1\textsuperscript{ère} espèce.}
\end{result}
\textcolor{Blue}{\textit{\ul{Question :} Ca peut se noter $det(\mathcal{F})$ avec $\mathcal{F}$ une famille de vecteurs ?}}


