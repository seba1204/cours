Soit $t\in\left[0,\pi\right]$.\\
$M(t)\in\Gamma_1$ est un point stationnaire si
$
  \left\{
  \begin{array}{rcl}
    x'(t) & = & 0 \\
    y'(t) & = & 0 \\
  \end{array}
  \right.
$.
\[
  \iff
  \left\{
  \begin{array}{rcl}
    2sin(t)\left[1+2cos(t)\right]                         & = & 0 \\
    4\left[1-cos(t)\right]\left[cos(t)+\frac{1}{2}\right] & = & 0 \\
  \end{array}
  \right.
\]

\[
  \iff
  \left\{
  \begin{array}{rcc}
    t & = & 0              \\
    cos(t) & = & -\frac{1}{2} \\
  \end{array}
  \right.
\]
\[
  \iff
  \left\{
  \begin{array}{rcc}
    t & = & 0              \\
    t & = & \frac{2\pi}{3} \\
  \end{array}
  \right.
\]
Ainsi :
\begin{result}
  $\Gamma_1$ possède 2 points stationnaires en $t=0$ et $t=\frac{2\pi}{3}$.
\end{result}
