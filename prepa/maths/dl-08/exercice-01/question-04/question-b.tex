Pour $t\in\mathbb{R}$, on a :
\[
  \overrightarrow{f}(t) =
  \left(
  \begin{array}{c}
      -2cos(t)-cos(2t) \\
      2sin(t)-sin(2t)  \\
    \end{array}
  \right)
  \text{et}
  \overrightarrow{f}'(t) =
  \left(
  \begin{array}{c}
      2sin(t)+2sin(2t) \\
      2cos(t)-2cos(2t) \\
    \end{array}
  \right)
\]
Définissons l'angle de relèvement $\alpha$ :
\[\forall t\in\mathbb{R}, \alpha(t) = \arctan{\left(\frac{y'(t)}{x'(t)}\right)}\]
et :
\[
  \forall t\in\mathbb{R},\gamma(t) = \frac{\alpha'(t)}{\left\|f'(t)\right\|}
\]
avec :
\[
  R = \frac{1}{\gamma}
\]
\[
  \alpha = arctan
  \left(2
  \frac{cos(t)-cos(2t)}{sin(t)+sin(2t)}
  \right)
\]
\[
  \iff
  \alpha' = \text{heu ça donne une grosse formule !}
\]
Ainsi, pour $t = \frac{\pi}{3}$, $\alpha' = \frac{1}{2}$.\\
De plus,
$\left\|f'(t)\right\| = \sqrt{2^2+\left(2\sqrt{3}\right)^2} = 4$
D'où
\begin{result}
  $R = 8$
\end{result}
