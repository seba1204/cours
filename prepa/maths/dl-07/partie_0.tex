\section{Questions préliminaires}
\begin{enumerate}
    \item \begin{enumerate}
              \item  Soit $\omega_0 = 1 + i 2\sqrt{2} \in \mathbb{C}$.
                    Restons en écriture algébrique.\\
                    Soit $Z = x + iy$ tel que $Z^{2}=\omega_0$.
                    \[
                        Z^{2} = x^{2} + 2ixy - y^{2}
                    \]
                    \[
                        \iff
                        \left\{
                        \begin{array}{r c l}
                            x^{2} + 2ixy - y^{2} & = & 1 + i 2\sqrt{2}               \\
                            \lvert Z^{2} \rvert  & = & \lvert 1 + i 2\sqrt{2} \rvert \\
                        \end{array}
                        \right.\newline
                    \]
                    \[
                        \iff
                        \left\{
                        \begin{array}{r c l}
                            x^{2} - y^{2} & = & 1         \\
                            2xy           & = & 2\sqrt{2} \\
                            x^{2} + y^{2} & = & 3         \\
                        \end{array}
                        \right.
                    \]
                    \[
                        \iff
                        \left\{
                        \begin{array}{r c l}
                            x^{2} & = & 2        \\
                            y^{2} & = & 1        \\
                            xy    & = & \sqrt{2} \\
                        \end{array}
                        \right.
                    \]
                    \[
                        \Rightarrow
                        \left\{
                        \begin{array}{r c l c r c l}
                            x & = & \sqrt{2}  & ET & y & = & 1  \\
                            x & = & -\sqrt{2} & ET & y & = & -1 \\
                        \end{array}
                        \right.
                    \]
                    D'où
                    \begin{result}
                        $
                            \left\{
                            \begin{array}{r c l}
                                \omega_1 & = & \sqrt{2} + i  \\
                                \omega_2 & = & -\sqrt{2} - i \\
                            \end{array}
                            \right.
                        $
                    \end{result}
              \item
                    Soit $\varphi = \arccos{(\frac{1}{3})}$\\ Passons en complexe, et à l'aide des formules d'Euler $cos(\varphi) = \frac{e^{i\varphi}+e^{-i\varphi}}{2}$, déterminons $arccos$ et $arctan$ en complexe.
                    % \begin{result}
                    %     D'où $
                    %         \varphi
                    %         = \arccos{(\frac{1}{3})}
                    %         = \arctan{(2\sqrt(2))}
                    %         = 2\arctan{(\frac{1}{\sqrt{2}})}
                    %     $
                    % \end{result}
          \end{enumerate}{}
    \item \begin{enumerate}
              \item
                    Une matrice carrée $Q$ othogonale est une matrice de taille
                    $n\times n$ avec $n\in\mathbb{N}$,
                    et telle que ses colonnes soient toutes de normes 1, et toutes othogonales entre elles.\\
                    i.e.
                    \begin{result}
                        $QQ^{T}=I_n$
                    \end{result}
              \item
                    Soit $Q\in \mathcal{M}_3(\mathbb{R})$ .\\
                    Si $Q$ est othogonale alors son endomorphisme canoniquement associé peut être :\\
                    $
                        \left\{
                        \text{une isométrie positive}\\
                        \text{une isométrie négative}\\
                        \right.
                    $\\
                    Il faut regarder $det(Q)$ pour différencier le type d'isométrie. Ainsi :

                    \begin{result}
                        $
                            \left\{
                            \begin{array}{r c l}
                                det(Q) = 1  & \Rightarrow{} & \text{isométrie positive} \\
                                det(Q) = -1 & \Rightarrow{} & \text{isométrie négative} \\
                            \end{array}
                            \right.
                        $
                    \end{result}
          \end{enumerate}
\end{enumerate}{}
