\section{Importation des données}
\begin{enumerate}
  \item \q{En vous inspirant du document }\il{Fichiers_csv.pdf}\q{ importer dans deux listes }\il{X}\q{ et }\il{Y}
        \q{les données issues de la MMT. On prendra soin de ne laisser dans ces listes que les données, c'est-à-dire que
          les titres seront enlevés.}


        \bigskip

        \begin{dinglist}{111}
          \item On commence comme toujours :
          \codeFromFile{section-02/q1-1.py}
          \item
          On crée un dossier \il{ressources} dans lequel on place le fichier \il{fichier_point.csv}
          \item
          On ouvre et lit ce fichier comme demandé :
          \codeFromFile{section-02/q1-2.py}
          Notons que tous les lignes contenant des chaînes de caractères vont faire échouer les lignes 8 ou 9, et donc
          seront passées.
        \end{dinglist}
  \item \q{Dans la mesure où on veut utiliser }\il{Numpy}\q{ pour travailler les tableaux, transformer ces listes en
          tableau }\il{numpy.ndarray}\q{.}
        \codeFromFile{section-02/q2.py}
\end{enumerate}
