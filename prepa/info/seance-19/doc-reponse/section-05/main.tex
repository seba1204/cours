\section{Meilleur que les meilleurs ?}
\q{Montrer que notre fonction }\il{simpson}\q{ donne le calcul de $J$ un
  meilleur résultat que la fonction }\il{quad}\q{de la bibliothèque}
\il{scipy.integrate}\q{ de python. Comment cela est-il possible d'après vous ?
  Quelles peuvent-être les limites de la méthode de Simpson ?}

\codeFromFileT{main.py}{section-05/q1.py}
Ce qui donne :

\[
  \left\{
  \begin{array}{rcl}
    J           & = & \il{0.20013802915686038}   \\
    \varepsilon & = & \il{9.167238679541269e-10} \\
  \end{array}
  \right.
\]

Ainsi, la méthode Simpson est plus précise que la méthode de résolution de scipy
(qui est précise à $10^{-9}$ près).
La limite de la méthode de Simpson est une précision de $10^{-12}$ max.
