\section{Décimal - Binaire}
\q{L'objectif est de créer un programme qui code un nombre entier décimal en binaire.}
\q{Ainsi, $255$ en décimal correspond à $10 0000 1101$ en binaire.}
\begin{enumerate}
  \item \q{Ecrire une fonction }
        \il{div2(x)}
        \q{ qui retourne la division entière par 2 d'un nombre entier et le reste de cette division.}
        \codeFromFile{section-03/q1.py}
  \item \q{Ecrire le programme qui met dans une pile les caractères représentant l'image binaire inversée du nombre décimal.}
        \codeFromFile{section-03/q2.py}
  \item \q{Faire afficher ce résultat.}
        \codeFromFile{section-03/q3.py}
  \item \q{Pour plus de lisibilité, un espace sera inséré tous les 4 caractères en partant du dernier. Modifier le programme précédent pour intégrer cette nouvelle exigence.}
        \codeFromFile{section-03/q4.py}
\end{enumerate}
