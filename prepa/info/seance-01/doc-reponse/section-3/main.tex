\section{Exemple 2}
On donne les commandes suivantes :
\begin{code}
  a = ["t", "o", "t", "o"]
  a.remove("o")
  c = a.remove("o")
  b = "toto"
  b.strip("o")    # supprime "o" s'il se trouve en debut ou fin de chaine
  d = b.strip("o")
\end{code}
\begin{enumerate}
  \item \q{Deviner le contenu des variables}
        \inlineCode{a}
        \q{, }
        \inlineCode{b}
        \q{, }
        \inlineCode{c}
        \q{ et }
        \inlineCode{d}
        \q{sans exécuter le code.}\\

        \inlineCode{a = "tt"}, \inlineCode{b = "tot"}, \inlineCode{c = null} et \inlineCode{d = null}.
\end{enumerate}


