\section{Exercice 3 : Suite de Syracuse}
\begin{enumerate}[(a)]
  \item \q{Déclarer la fonction} \inlineCode{S} \q{en utilisant reste et quotient dans les nombres entiers.}
        \codeFromFile{section-6/qa.py}
  \item \q{Afficher la liste des valeurs $u_p = S^p(c)$ pour $p$ allant de $0$ à $+\infty$ et en s'arrêtant dès que $u_p=1$.}
        \codeFromFile{section-6/qb.py}
  \item \q{Tester pour $x$ dans la liste }\inlineCode{[2, 7, 19, 23, 29]}.
        \codeFromFile{section-6/qc.py}
  \item \q{Trouver les entiers $x$ compris entre $3$ et $300$ pour lequel le vol (c'est-à-dire la longueur de la suite pour obtenir 1) est le plus long.}
        \codeFromFile{section-6/qd.py}
        \begin{result}
          On trouve que le vol le plus long est $v=128$, et est atteint pour $x\in\{231, 235\}$.
        \end{result}
  \item \q{Calculer l'intersection entre deux suites de syracuse en renvoyant la longueur de l'intersection et les derniers éléments où les deux suites sont différentes. En cas d'inclusion, on renvoie le complexe $1j$ pour indiquer qu'une suite est incluse dans l'autre.}
        \codeFromFile{section-6/qe.py}
        On trouve :
        \begin{result}
          $
            \begin{array}{l}
              interSyracuse(7, 320) = (9, 80, 13) \\
              interSyracuse(7, 230) = 1j
            \end{array}
          $
        \end{result}
\end{enumerate}

